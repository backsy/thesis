% This is the Reed College LaTeX thesis template. Most of the work
% for the document class was done by Sam Noble (SN), as well as this
% template. Later comments etc. by Ben Salzberg (BTS). Additional
% restructuring and APA support by Jess Youngberg (JY).
% Your comments and suggestions are more than welcome; please email
% them to cus@reed.edu
%
% See http://web.reed.edu/cis/help/latex.html for help. There are a
% great bunch of help pages there, with notes on
% getting started, bibtex, etc. Go there and read it if you're not
% already familiar with LaTeX.
%
% Any line that starts with a percent symbol is a comment.
% They won't show up in the document, and are useful for notes
% to yourself and explaining commands.
% Commenting also removes a line from the document;
% very handy for troubleshooting problems. -BTS

% As far as I know, this follows the requirements laid out in
% the 2002-2003 Senior Handbook. Ask a librarian to check the
% document before binding. -SN

%%
%% Preamble
%%
% \documentclass{<something>} must begin each LaTeX document
\documentclass[12pt,twoside]{reedthesis}
% Packages are extensions to the basic LaTeX functions. Whatever you
% want to typeset, there is probably a package out there for it.
% Chemistry (chemtex), screenplays, you name it.
% Check out CTAN to see: http://www.ctan.org/
%%
\usepackage{graphicx,latexsym}
\usepackage{amsmath}
\usepackage{amssymb,amsthm}
\usepackage{longtable,booktabs,setspace}
\usepackage{chemarr} %% Useful for one reaction arrow, useless if you're not a chem major
\usepackage[hyphens]{url}
% Added by CII
\usepackage{hyperref}
\usepackage{lmodern}
\usepackage{float}
\floatplacement{figure}{H}
% End of CII addition
\usepackage{rotating}

% Next line commented out by CII
%%% \usepackage{natbib}
% Comment out the natbib line above and uncomment the following two lines to use the new
% biblatex-chicago style, for Chicago A. Also make some changes at the end where the
% bibliography is included.
%\usepackage{biblatex-chicago}
%\bibliography{thesis}


% Added by CII (Thanks, Hadley!)
% Use ref for internal links
\renewcommand{\hyperref}[2][???]{\autoref{#1}}
\def\chapterautorefname{Chapter}
\def\sectionautorefname{Section}
\def\subsectionautorefname{Subsection}
% End of CII addition

% Added by CII
\usepackage{caption}
\captionsetup{width=5in}
% End of CII addition

% \usepackage{times} % other fonts are available like times, bookman, charter, palatino

% Syntax highlighting #22
  \usepackage{color}
  \usepackage{fancyvrb}
  \newcommand{\VerbBar}{|}
  \newcommand{\VERB}{\Verb[commandchars=\\\{\}]}
  \DefineVerbatimEnvironment{Highlighting}{Verbatim}{commandchars=\\\{\}}
  % Add ',fontsize=\small' for more characters per line
  \usepackage{framed}
  \definecolor{shadecolor}{RGB}{248,248,248}
  \newenvironment{Shaded}{\begin{snugshade}}{\end{snugshade}}
  \newcommand{\KeywordTok}[1]{\textcolor[rgb]{0.13,0.29,0.53}{\textbf{#1}}}
  \newcommand{\DataTypeTok}[1]{\textcolor[rgb]{0.13,0.29,0.53}{#1}}
  \newcommand{\DecValTok}[1]{\textcolor[rgb]{0.00,0.00,0.81}{#1}}
  \newcommand{\BaseNTok}[1]{\textcolor[rgb]{0.00,0.00,0.81}{#1}}
  \newcommand{\FloatTok}[1]{\textcolor[rgb]{0.00,0.00,0.81}{#1}}
  \newcommand{\ConstantTok}[1]{\textcolor[rgb]{0.00,0.00,0.00}{#1}}
  \newcommand{\CharTok}[1]{\textcolor[rgb]{0.31,0.60,0.02}{#1}}
  \newcommand{\SpecialCharTok}[1]{\textcolor[rgb]{0.00,0.00,0.00}{#1}}
  \newcommand{\StringTok}[1]{\textcolor[rgb]{0.31,0.60,0.02}{#1}}
  \newcommand{\VerbatimStringTok}[1]{\textcolor[rgb]{0.31,0.60,0.02}{#1}}
  \newcommand{\SpecialStringTok}[1]{\textcolor[rgb]{0.31,0.60,0.02}{#1}}
  \newcommand{\ImportTok}[1]{#1}
  \newcommand{\CommentTok}[1]{\textcolor[rgb]{0.56,0.35,0.01}{\textit{#1}}}
  \newcommand{\DocumentationTok}[1]{\textcolor[rgb]{0.56,0.35,0.01}{\textbf{\textit{#1}}}}
  \newcommand{\AnnotationTok}[1]{\textcolor[rgb]{0.56,0.35,0.01}{\textbf{\textit{#1}}}}
  \newcommand{\CommentVarTok}[1]{\textcolor[rgb]{0.56,0.35,0.01}{\textbf{\textit{#1}}}}
  \newcommand{\OtherTok}[1]{\textcolor[rgb]{0.56,0.35,0.01}{#1}}
  \newcommand{\FunctionTok}[1]{\textcolor[rgb]{0.00,0.00,0.00}{#1}}
  \newcommand{\VariableTok}[1]{\textcolor[rgb]{0.00,0.00,0.00}{#1}}
  \newcommand{\ControlFlowTok}[1]{\textcolor[rgb]{0.13,0.29,0.53}{\textbf{#1}}}
  \newcommand{\OperatorTok}[1]{\textcolor[rgb]{0.81,0.36,0.00}{\textbf{#1}}}
  \newcommand{\BuiltInTok}[1]{#1}
  \newcommand{\ExtensionTok}[1]{#1}
  \newcommand{\PreprocessorTok}[1]{\textcolor[rgb]{0.56,0.35,0.01}{\textit{#1}}}
  \newcommand{\AttributeTok}[1]{\textcolor[rgb]{0.77,0.63,0.00}{#1}}
  \newcommand{\RegionMarkerTok}[1]{#1}
  \newcommand{\InformationTok}[1]{\textcolor[rgb]{0.56,0.35,0.01}{\textbf{\textit{#1}}}}
  \newcommand{\WarningTok}[1]{\textcolor[rgb]{0.56,0.35,0.01}{\textbf{\textit{#1}}}}
  \newcommand{\AlertTok}[1]{\textcolor[rgb]{0.94,0.16,0.16}{#1}}
  \newcommand{\ErrorTok}[1]{\textcolor[rgb]{0.64,0.00,0.00}{\textbf{#1}}}
  \newcommand{\NormalTok}[1]{#1}

% To pass between YAML and LaTeX the dollar signs are added by CII
\title{Wind estimation for fuel management}
\author{Karl Uibo}
% The month and year that you submit your FINAL draft TO THE LIBRARY (May or December)
\date{May 2018}
\division{Information Technologies}
\advisor{Olev Märtens}
\institution{Tallinn University of Technology}
\degree{Master of Arts}
%If you have two advisors for some reason, you can use the following
% Uncommented out by CII
% End of CII addition

%%% Remember to use the correct department!
\department{Thomas Johann Seebeck Department of Electronics}
% if you're writing a thesis in an interdisciplinary major,
% uncomment the line below and change the text as appropriate.
% check the Senior Handbook if unsure.
%\thedivisionof{The Established Interdisciplinary Committee for}
% if you want the approval page to say "Approved for the Committee",
% uncomment the next line
%\approvedforthe{Committee}

% Added by CII
%%% Copied from knitr
%% maxwidth is the original width if it's less than linewidth
%% otherwise use linewidth (to make sure the graphics do not exceed the margin)
\makeatletter
\def\maxwidth{ %
  \ifdim\Gin@nat@width>\linewidth
    \linewidth
  \else
    \Gin@nat@width
  \fi
}
\makeatother

\renewcommand{\contentsname}{Table of Contents}
% End of CII addition

\setlength{\parskip}{0pt}

% Added by CII

\providecommand{\tightlist}{%
  \setlength{\itemsep}{0pt}\setlength{\parskip}{0pt}}

\Originality{
I hereby certify that I am the sole author of this thesis. All the used
materials, references to the literature and the work of others have been
referred to. This thesis has not been presented for examination anywhere
else.

~

Author:
}

\Annotatsioon{
Lõputöö on kirjutatud Inglise keeles ning sisaldab teksti xx leheküljel,
xx peatükki, xx joonist, xx tabelit.
}

\Abstract{
This thesis is written in and is pages long, including chapters, figures
and tables.
}

\Abbrvs{
Abbreviations of stuff and such. Kable it i guess.
}

% End of CII addition
%%
%% End Preamble
%%
%

\usepackage{amsthm}
\newtheorem{theorem}{Theorem}[chapter]
\newtheorem{lemma}{Lemma}[chapter]
\newtheorem{corollary}{Corollary}[chapter]
\newtheorem{proposition}{Proposition}[chapter]
\newtheorem{conjecture}{Conjecture}[chapter]
\theoremstyle{definition}
\newtheorem{definition}{Definition}[chapter]
\theoremstyle{definition}
\newtheorem{example}{Example}[chapter]
\theoremstyle{definition}
\newtheorem{exercise}{Exercise}[chapter]
\theoremstyle{remark}
\newtheorem*{remark}{Remark}
\newtheorem*{solution}{Solution}
\begin{document}

% Everything below added by CII
  \maketitle

\frontmatter % this stuff will be roman-numbered
\pagestyle{empty} % this removes page numbers from the frontmatter
  \begin{originality}
    I hereby certify that I am the sole author of this thesis. All the used
    materials, references to the literature and the work of others have been
    referred to. This thesis has not been presented for examination anywhere
    else.
    
    ~
    
    Author:
  \end{originality}
  \begin{abstract}
    This thesis is written in and is pages long, including chapters, figures
    and tables.
  \end{abstract}
  \begin{annotatsioon}
    Lõputöö on kirjutatud Inglise keeles ning sisaldab teksti xx leheküljel,
    xx peatükki, xx joonist, xx tabelit.
  \end{annotatsioon}
  \begin{abbrvs}
    Abbreviations of stuff and such. Kable it i guess.
  \end{abbrvs}
  \hypersetup{linkcolor=black}
  \setcounter{tocdepth}{2}
  \tableofcontents

  \listoffigures

  \listoftables

\mainmatter % here the regular arabic numbering starts
\pagestyle{fancyplain} % turns page numbering back on

\chapter*{Introduction}\label{introduction}
\addcontentsline{toc}{chapter}{Introduction}

The following thesis looks at the flight logs of Eli Ltd. to assess the
fuel economy and the information given to the user. To be written when
rest of the paper is finished.

\chapter{Problem}\label{problem}

First chapters contents.

\chapter{State of the art}\label{state}

\section{Math}\label{math}

\TeX~is the best way to typeset mathematics. Donald Knuth designed
\TeX~when he got frustrated at how long it was taking the typesetters to
finish his book, which contained a lot of mathematics. One nice feature
of \emph{R Markdown} is its ability to read LaTeX code directly.

If you are doing a thesis that will involve lots of math, you will want
to read the following section which has been commented out. If you're
not going to use math, skip over or delete this next commented section.

\section{Chemistry 101: Symbols}\label{chemistry-101-symbols}

Chemical formulas will look best if they are not italicized. Get around
math mode's automatic italicizing in LaTeX by using the argument
\texttt{\$\textbackslash{}mathrm\{formula\ here\}\$}, with your formula
inside the curly brackets. (Notice the use of the backticks here which
enclose text that acts as code.)

So, \(\mathrm{Fe_2^{2+}Cr_2O_4}\) is written
\texttt{\$\textbackslash{}mathrm\{Fe\_2\^{}\{2+\}Cr\_2O\_4\}\$}.

\noindent Exponent or Superscript: \(\mathrm{O^-}\)

\noindent Subscript: \(\mathrm{CH_4}\)

To stack numbers or letters as in \(\mathrm{Fe_2^{2+}}\), the subscript
is defined first, and then the superscript is defined.

\noindent Bullet: CuCl \(\bullet\) \(\mathrm{7H_{2}O}\)

\noindent Delta: \(\Delta\)

\noindent Reaction Arrows: \(\longrightarrow\) or
\(\xrightarrow{solution}\)

\noindent Resonance Arrows: \(\leftrightarrow\)

\noindent Reversible Reaction Arrows: \(\rightleftharpoons\)

\subsection{Typesetting reactions}\label{typesetting-reactions}

You may wish to put your reaction in an equation environment, which
means that LaTeX will place the reaction where it fits and will number
the equations for you.
\begin{equation}
  \mathrm{C_6H_{12}O_6  + 6O_2} \longrightarrow \mathrm{6CO_2 + 6H_2O}
  \label{eq:reaction}
\end{equation}
We can reference this combustion of glucose reaction via Equation
\eqref{eq:reaction}.

\subsection{Other examples of
reactions}\label{other-examples-of-reactions}

\(\mathrm{NH_4Cl_{(s)}}\) \(\rightleftharpoons\)
\(\mathrm{NH_{3(g)}+HCl_{(g)}}\)

\noindent \(\mathrm{MeCH_2Br + Mg}\) \(\xrightarrow[below]{above}\)
\(\mathrm{MeCH_2\bullet Mg \bullet Br}\)

\section{Physics}\label{physics}

Many of the symbols you will need can be found on the math page
\url{http://web.reed.edu/cis/help/latex/math.html} and the Comprehensive
LaTeX Symbol Guide
(\url{http://mirror.utexas.edu/ctan/info/symbols/comprehensive/symbols-letter.pdf}).

\section{Biology}\label{biology}

You will probably find the resources at
\url{http://www.lecb.ncifcrf.gov/~toms/latex.html} helpful, particularly
the links to bsts for various journals. You may also be interested in
TeXShade for nucleotide typesetting
(\url{http://homepages.uni-tuebingen.de/beitz/txe.html}). Be sure to
read the proceeding chapter on graphics and tables.

\chapter{Database generation}\label{dbs}

How i made the database i did and why.

\chapter{Modelling stuff}\label{modelling-stuff}

Eeem.

\chapter{The results of it all}\label{the-results-of-it-all}

The stuff i wanted to do and did.

\chapter*{Conclusion}\label{conclusion}
\addcontentsline{toc}{chapter}{Conclusion}

Welp, thats it.

\appendix

\chapter{The First Appendix}\label{the-first-appendix}

This first appendix includes all of the R chunks of code that were
hidden throughout the document (using the \texttt{include\ =\ FALSE}
chunk tag) to help with readibility and/or setup.

\textbf{In the main Rmd file}
\begin{Shaded}
\begin{Highlighting}[]
\CommentTok{# This chunk ensures that the thesisdown package is}
\CommentTok{# installed and loaded. This thesisdown package includes}
\CommentTok{# the template files for the thesis.}
\ControlFlowTok{if}\NormalTok{(}\OperatorTok{!}\KeywordTok{require}\NormalTok{(devtools))}
  \KeywordTok{install.packages}\NormalTok{(}\StringTok{"devtools"}\NormalTok{, }\DataTypeTok{repos =} \StringTok{"http://cran.rstudio.com"}\NormalTok{)}
\CommentTok{# Probably should install this from my own repo!!}
\ControlFlowTok{if}\NormalTok{(}\OperatorTok{!}\KeywordTok{require}\NormalTok{(thesisdown))}
\NormalTok{  devtools}\OperatorTok{::}\KeywordTok{install_github}\NormalTok{(}\StringTok{"ismayc/thesisdown"}\NormalTok{)}
\KeywordTok{library}\NormalTok{(thesisdown)}
\end{Highlighting}
\end{Shaded}
\textbf{In Chapter \ref{ref-labels}:}

\backmatter

\chapter*{References}\label{references}
\addcontentsline{toc}{chapter}{References}

\markboth{References}{References}

\noindent

\setlength{\parindent}{-0.20in} \setlength{\leftskip}{0.20in}
\setlength{\parskip}{8pt}

\hypertarget{refs}{}
\hypertarget{ref-angel2000}{}
Angel, E. (2000). \emph{Interactive computer graphics : A top-down
approach with opengl}. Boston, MA: Addison Wesley Longman.

\hypertarget{ref-angel2001}{}
Angel, E. (2001a). \emph{Batch-file computer graphics : A bottom-up
approach with quicktime}. Boston, MA: Wesley Addison Longman.

\hypertarget{ref-angel2002a}{}
Angel, E. (2001b). \emph{Test second book by angel}. Boston, MA: Wesley
Addison Longman.


% Index?

\end{document}
